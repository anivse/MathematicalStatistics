\section{Теория}

\subsection{Рассматриваемые распределения}

Плотности вероятности рассматриваемых распределений: 

\begin{itemize}
	\item Нормальное распределение
	
	\begin{equation}
		N(x, 0, 0) = \tfrac{1}{ \sqrt{2\pi}}e^{-\tfrac{x^2}{2}}
	\end{equation}
	
	\item Распределение Коши
	
	\begin{equation}
		C(x, 0, 1) = \dfrac{1}{\pi}\dfrac{1}{1 + x^2}
	\end{equation}
	
	\item Распределение Лапласа
	
	\begin{equation}
		L(x, 0, \frac{1}{\sqrt{2}}) =\tfrac{1}{\sqrt{2}}e^{-\sqrt{2}|x|}
	\end{equation}
	
	\item Распределение Пуассона
	
	\begin{equation}
		P(k, 10) = \tfrac{10^k}{k!}e^{-10}
	\end{equation}
	
	\item Равномерное распределение
	
	\begin{equation}
		U(x, -\sqrt{3}, \sqrt{3}) =
		\begin{cases} 
			\;\; \dfrac{1}{2\sqrt{3}} \;\;\;\; \text{при} \;\;\;\; |x| \leq \sqrt{3}\\
			\,\,\,\:\;\; 0 \;\;\;\;\;\;\; \text{при} \;\;\;\; |x| > \sqrt{3}
		\end{cases}
	\end{equation}
	
\end{itemize}

\subsection{Гистограмма}

Множество значений, которое может принимать элемент выборки разбивается на несколько интервалов. Чаще всего эти интервалы берутся одинаковыми (но не обязательно). Данные интервалы откладываются на горизонтальной оси, затем над каждым рисуется прямоугольник. Если все интервалы одинаковые, то высота каждого прямоугольника пропорциональна числу элементов выборки, попадающих в соответствующий интервал. Если интервалы разные, то высота прямоугольника выбирается так, чтобы его площадь была пропорциональна числу элементов выборки, попавших в данный интервал.

\subsection{Вариационный ряд}

Вариационный ряд -- последовательность элементов выборки, расположенных в неубывающем порядке. Одинаковые элементы повторяются.

\subsection{Выборочные числовые характеристики}

\subsubsection{Характеристики положения}

\begin{itemize}
	\item Выборочное среднее
	
	\begin{equation} \label{eq:mean}
		\overline{x} = \tfrac{1}{n}\sum\limits_{i=1}^n x_i
	\end{equation}
	
	\item Выборочная медиана
	
	\begin{equation} \label{eq:med}
		med \: x = 
		\begin{cases} 
			\;\;\;\;\;\;\; x_{(l+1)} \:\;\;\;\;\;\;\;\;\; \text{при} \;\;\;\; n = 2l + 1\\
			\;\; \dfrac{x_{(l)} + x_{(l+1)}}{2} \;\;\;\; \text{при} \;\;\;\; n = 2l
		\end{cases}
	\end{equation}
	
	\item Полусумма экстремальных выборочных элементов
	
	\begin{equation} \label{eq:zR}
		z_R = \dfrac{x_{(1)} + x_{(n)}}{2}
	\end{equation}
	
	\item Полусумма квартилей
	
	Выборочная квартиль $z_p$ порядка $p$ определяется формулой
	
	\begin{equation}
		z_p =
		\begin{cases}
			\;\; x_{([np]+1)} \;\;\;\; \text{при} \;\; np \;\; \text{дробном},\\
			\;\;\;\;\; x_{(np)} \,\:\;\;\;\;\; \text{при} \;\; np \;\; \text{целом}.
		\end{cases}
	\end{equation}
	
	Полусумма квартилей
	
	\begin{equation} \label{eq:zQ}
		z_Q = \dfrac{z_{1/4} + z_{3/4}}{2}
	\end{equation}
	
	\item Усечённое среднее
	
	\begin{equation} \label{eq:zTr}
		z_{tr} = \tfrac{1}{n-2r}\sum\limits_{i=r+1}^{n-r} x_{(i)}, \;\;\;\; r \approx \dfrac{n}{4}
	\end{equation}
	
\end{itemize}

\subsubsection{Характеристики рассеяния}

Выборочная дисперсия

\begin{equation}
	D = \tfrac{1}{n}\sum\limits_{i=1}^{n} (x_i - \overline{x})^2
\end{equation}


