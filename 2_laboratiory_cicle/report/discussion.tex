\section{Обсуждение}

\subsection{Выборочные коэффициенты корреляции}

Для двумерного нормального распределения и смеси распределений дисперсии выборочных коэффициентов корреляции упорядочены в основном следующим образом: $r < r_s < r_q$

\subsection{Эллипсы рассеивания}

Процент попавших элементов выборки в эллипс рассеивания (95\%-ная доверительная область) примерно равен его теоретическому значению (95\%).

\subsection{Оценки коэффициентов линейной регрессии}

На основе графиков можно сделать вывод о том, что критерий наименьших модулей оценивает коэффициенты линейной регрессии точнее критерия наименьших квадратов как на выборке без возмущений, так и на выборке с возмущениями, причем отклонение оценок, полученных с помощью критерия наименьших квадратов, в случае возмущений более существенное. 
Оценка относительного отклонения \eqref{tab:relative_deviation} параметров $a$ и $b$ говорит, что в целом метод наименьших модулей более устойчив к редким возмущениями чем метод наименьших квадратов.


\subsection{Проверка гипотезы о законе распределения генеральной совокупности методом $\chi^2$ }

Гипотеза $H_0$ о норамальном законе распределения оказалась принята для выборки нормального распределения $N(x, 0, 1)$ размером 100 элементов на уровне значимости $\alpha = 0.05$. Также данная гипотеза оказалась принята и для выборок равномерного распределения и распределения Лапласа размером 20 элементов. \\
Поскольку статистика критерия хи-квадрат асимптотически приближается к плотности функции распределения случайной величины, то при маленьком объеме выборки достаточно трудно судить о достоверности принятия гипотезы $H_0$.
Мы увидели, что гипотезы для равномерного закона и закона распределения Лапласа на маленьких выборках были приняты, но для большей уверенности следует проверить статистику критерия на большей выборке.

\subsection{Доверительные интервалы для параметров нормального распределения}

Генеральные характеристики $m = 0$ и $\sigma = 1$ оказались полностью покрыты классическими и ассимптотическими доверительными интервалами для обеих рассматриваемых выборок. \\
Также можно заметить, что с увеличением объема выборки размер доверительного интервала уменьшается, то есть интервалы становятся более точными.\\
Для асимптотической оценки длины границ твинов в среднем получились больше чем для классической оценки. \\
  

