\section{Обсуждение}

\subsection{Оценки исходной выборки} 

На основе полученных результатов можно сделать вывод, что верхние и нижние вершины оценок $\bm{J}_1$ совпадают с границами отображения на рис.3.

\subsection{Мода и максимальная клика выборки} 

Полученная мода входит в большую часть элементов выборки, что свидетельствуют о невысокой степени несовместности выборки.

\subsection{Варьирование неопределенности изменений}

Полученная оценка постоянной $\beta$ получилась очень близка к вычисленной ранее моде. \\
Величина однородного расширения интервалов невелика, что свидетельствует о невысокой степени несовместности выборки. 

\subsection{Коэффициент Жакара и относительная ширина моды}

Отрицательности коэффициента Жакара свидетельствует о несовместности выборки, а модуль - о степени несовместности. В данном случае, можно сделать вывод, что выборка несовместна, но степень несовместности невелика. \\

Величина относительной ширины моды составляет менее 2\% внешней оценки выборки $\bm{X}_1$.