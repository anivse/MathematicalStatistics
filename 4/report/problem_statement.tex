\section{Постановка задачи}

Имеется выборка данных с интервальной неопределенностью. Число отсчетов в выборке равно 200. Используется модель данных с  уравновешенным интервалом погрешности. \\

$\bm{x} = \stackrel{\circ}{x} + \bm{\epsilon}$; \quad $\bm{\epsilon} = [-\epsilon, \epsilon]$  для некоторого $\epsilon >0 $, \\

Здесь $\stackrel{\circ}{x}$ -- данные некоторого прибора, $\epsilon = 10 ^ {-4}$ -- погрешность прибора.

Нужно иллюстрировать данные выборки, построить диаграмму рассеяния, построить линейную регрессионную зависимость варьированием неопределенности изменений (без сужения и с расширением и сужением интервалов),  произвести анализ регрессионных остатков, построить информационное множество по модели, проиллюстрирвоать коридор совместных зависимостей, построить прогноз вне области данных. 

