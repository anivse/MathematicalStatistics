\section{Обсуждение}

\subsection{Гистограмма и график плотности распределения}

Исходя из результатов можно сделать вывод, что чем мощнее выборка, тем ближе ее гистограмма к графику плотности вероятности распределения, по которому данная выборка сгенерирована и тем лучше по ней можно определить характер распределения. \\
На основе выборок с маленьким размером (n=10) определить характер распределения почти невозможно. Например при n=10 невозможно отличить гистограммы равномерного распределения и распределения Пуассона. \\
Также на выборках с маленьким размером чаще наблюдаются всплески гистограмм (особенно у распределений Коши и Лапласа). 

\subsection{Характеристики положения и рассеяния}

Полученные результаты дисперсий характеристик рассеяния распределения Коши являются очень большими при всех размерах выборки. Можно сделать вывод о том, что это является следствием большого количества выбросов, которые можно также наблюдать на гистограммах. 

\subsection{Боксплот Тьюки}

Боксплоты Тьюки позволяют наглядно наблюдать характеристики распределений (например большое количество выбросов распределения Коши и отсутствия выбросов у равномерного распределения).

\subsection{Доля выбросов}

Исходя из полученых результатов мы получили что для всех распределений (за исключением нормального распределения с размером выборки 20) доли выбросов, полученные практически и теоретически приближенно равны. \\
Экспериментально и теоретически подтверждено большое количество выбросов распределения Коши, замеченное при выолнении предыдущих заданий и отсутствие выбросов у равномерного распределения.

\subsection{Эмпирическая функция распределения}

На основе полученных результатов можно сделать вывод о том, что чем больше размер выборки, тем лучше эмперическая функция распределения приближает теоретическую функцию распределения. Наибольшие отклонения при n=100 наблюдаются у распределения Пуассона.  

\subsection{Ядерные оценки плотности распределения}

На основе полученных результатов можно сделать вывод, что при увеличении размера размера выборки при всех h ядерная оценка начинает лучше приближать плотность вероятности. \\
Также можно сделать вывод, что параметр h в ядерной оценке лучше подбирать в зависимости от характера распределения. Параметр $ h = \dfrac{h_n}{2}$ лучше брать для распределений Коши и Лапласа. $h = h_n$ лучше всего походит для нормального распределения, а $ h = 2h_n $ - для распределений Пуассона и равномерного. \\
Также можно отметить, что при увеличении параметра h ядерной оценки уменьшается количество изменений знака производной. При $h=2h_n$ функция становится унимодальной. 