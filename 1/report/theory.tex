\section{Теория}

\subsection{Рассматриваемые распределения}

Плотности вероятности рассматриваемых распределений: 

\begin{itemize}
	\item Нормальное распределение
	
	\begin{equation}
		N(x, 0, 0) = \tfrac{1}{ \sqrt{2\pi}}e^{-\tfrac{x^2}{2}}
	\end{equation}
	
	\item Распределение Коши
	
	\begin{equation}
		C(x, 0, 1) = \dfrac{1}{\pi}\dfrac{1}{1 + x^2}
	\end{equation}
	
	\item Распределение Лапласа
	
	\begin{equation}
		L(x, 0, \frac{1}{\sqrt{2}}) =\tfrac{1}{\sqrt{2}}e^{-\sqrt{2}|x|}
	\end{equation}
	
	\item Распределение Пуассона
	
	\begin{equation}
		P(k, 10) = \tfrac{10^k}{k!}e^{-10}
	\end{equation}
	
	\item Равномерное распределение
	
	\begin{equation}
		U(x, -\sqrt{3}, \sqrt{3}) =
		\begin{cases} 
			\;\; \dfrac{1}{2\sqrt{3}} \;\;\;\; \text{при} \;\;\;\; |x| \leq \sqrt{3}\\
			\,\,\,\:\;\; 0 \;\;\;\;\;\;\; \text{при} \;\;\;\; |x| > \sqrt{3}
		\end{cases}
	\end{equation}
	
\end{itemize}

\subsection{Гистограмма}

Множество значений, которое может принимать элемент выборки разбивается на несколько интервалов. Чаще всего эти интервалы берутся одинаковыми (но не обязательно). Данные интервалы откладываются на горизонтальной оси, затем над каждым рисуется прямоугольник. Если все интервалы одинаковые, то высота каждого прямоугольника пропорциональна числу элементов выборки, попадающих в соответствующий интервал. Если интервалы разные, то высота прямоугольника выбирается так, чтобы его площадь была пропорциональна числу элементов выборки, попавших в данный интервал. \cite{s:hist}.

\subsection{Вариационный ряд}

Вариационный ряд -- последовательность элементов выборки, расположенных в неубывающем порядке. Одинаковые элементы повторяются. \cite[с. 409]{b:probSectMath}

\subsection{Выборочные числовые характеристики}
\cite[с. 409]{b:probSectMath}

\subsubsection{Характеристики положения}

\begin{itemize}
	\item Выборочное среднее
	
	\begin{equation} \label{eq:mean}
		\overline{x} = \tfrac{1}{n}\sum\limits_{i=1}^n x_i
	\end{equation}
	
	\item Выборочная медиана
	
	\begin{equation} \label{eq:med}
		med \: x = 
		\begin{cases} 
			\;\;\;\;\;\;\; x_{(l+1)} \:\;\;\;\;\;\;\;\;\; \text{при} \;\;\;\; n = 2l + 1\\
			\;\; \dfrac{x_{(l)} + x_{(l+1)}}{2} \;\;\;\; \text{при} \;\;\;\; n = 2l
		\end{cases}
	\end{equation}
	
	\item Полусумма экстремальных выборочных элементов
	
	\begin{equation} \label{eq:zR}
		z_R = \dfrac{x_{(1)} + x_{(n)}}{2}
	\end{equation}
	
	\item Полусумма квартилей
	
	Выборочная квартиль $z_p$ порядка $p$ определяется формулой
	
	\begin{equation}
		z_p =
		\begin{cases}
			\;\; x_{([np]+1)} \;\;\;\; \text{при} \;\; np \;\; \text{дробном},\\
			\;\;\;\;\; x_{(np)} \,\:\;\;\;\;\; \text{при} \;\; np \;\; \text{целом}.
		\end{cases}
	\end{equation}
	
	Полусумма квартилей
	
	\begin{equation} \label{eq:zQ}
		z_Q = \dfrac{z_{1/4} + z_{3/4}}{2}
	\end{equation}
	
	\item Усечённое среднее
	
	\begin{equation} \label{eq:zTr}
		z_{tr} = \tfrac{1}{n-2r}\sum\limits_{i=r+1}^{n-r} x_{(i)}, \;\;\;\; r \approx \dfrac{n}{4}
	\end{equation}
	
\end{itemize}

\subsubsection{Характеристики рассеяния}

Выборочная дисперсия

\begin{equation}
	D = \tfrac{1}{n}\sum\limits_{i=1}^{n} (x_i - \overline{x})^2
\end{equation}

\subsection{Боксплот Тьюки}

\subsubsection{Построение}

Границы ящика -- первый и третий квартили, линия в середине ящика -- медиана. Концы усов -- края статистически значимой выборки (без выбросов). Длина "усов": 

\begin{equation} \label{eq:boundBoxplot}
	X_1 = Q_1 - \dfrac{3}{2}(Q_3 - Q_1), \;\; X_2 = Q_3 + \dfrac{3}{2}(Q_3 - Q_1),
\end{equation}

где $X_1$ --- нижняя граница уса, $X_2$ --- верхняя граница уса, $Q_1$ --- первый квартиль, $Q_3$ --- третий квартиль.

Данные, выходящие за границы усов (выбросы), отображаются на графике в виде маленьких кружков.\cite{s:boxplot}

\subsection{Теоретическая вероятность выбросов}

Можно вычислить теоретические первый и третий квартили распределений ($Q_1^\text{т}$ и $Q_3^\text{т}$ соответственно). По формуле \eqref{eq:boundBoxplot} можно вычислить теоретические нижнюю и верхнюю границы уса ($X_1^\text{т}$ и $X_2^\text{т}$ соответственно). Выбросами считаются величины $x$, такие что:

\begin{equation}
	\left[ 
	\begin{gathered} 
		x < X_1^\text{т}\\ 
		x > X_2^\text{т}\\ 
	\end{gathered} 
	\right.
\end{equation}

Теоретическая вероятность выбросов для непрерывных распределений

\begin{equation} \label{eq:probTheorCont}
	P_\text{в}^\text{т} = P(x < X_1^\text{т}) + P(x > X_2^\text{т}) = F(X_1^\text{т}) + \Big(1 - F(X_2^\text{т})\Big),
\end{equation}

где $F(X) = P(x \le X)$ - функция распределения.\\

Теоретическая вероятность выбросов для дискретных распределений

\begin{equation} \label{eq:probTheorDisc}
	P_\text{в}^\text{т} = P(x < X_1^\text{т}) + P(x > X_2^\text{т}) = \Big(F(X_1^\text{т}) - P(x = X_1^\text{т})\Big) + \Big(1 - F(X_2^\text{т})\Big),
\end{equation}

где $F(X) = P(x \le X)$ - функция распределения.

\subsection{Эмпирическая функция распределения}

\subsubsection{Статистический ряд}

Статистическим рядом называется последовательность различных элементов выборки $z_1, z_2, \, ... \: , z_k$, расположенных в возрастающем порядке с указанием частот $n_1, n_2, \, ... \: , n_k$, с которыми эти элементы содержатся в выборке.

Статистический ряд обычно записывается в виде таблицы

\begin{table}[h!]
	\begin{center}
		\begin{tabular}{|c|c|c|c|c|}
			\hline
			$z$ & $z_1$ & $z_2$ & $...$ & $z_k$ \\
			\hline
			$n$ & $n_1$ & $n_2$ & $...$ & $n_k$ \\
			\hline
		\end{tabular}
	\end{center}
	\caption{Статистический ряд}
\end{table} 

\subsubsection{Определение}

Эмпирической (выборочной) функцией распределения (э. ф. р.) называется относительная частота события $X < x$, полученная по данной выборке:

\begin{equation}
	F_n^*(x) = P^*(X < x).
\end{equation}

\subsubsection{Описание}

Для получения относительной частоты $P^*(X < x)$ просуммируем в статистическом ряде, построенном по данной выборке, все частоты $n_i$, для которых элементы $z_i$ статистического ряда меньше $x$. Тогда $P^*(X < x) = \tfrac{1}{n}\sum\limits_{z_i < x}n_i$. Получаем

\begin{equation}
	F^*(x) = \tfrac{1}{n}\sum\limits_{z_i < x}n_i.
\end{equation}

$F^*(x)$ --- функция распределения дискретной случайной величины $X^*$, заданной таблицей распределения

\begin{table}[h!]
	\begin{center}
		\begin{tabular}{|c|c|c|c|c|}
			\hline
			$X^*$ & $z_1$ & $z_2$ & $...$ & $z_k$ \\
			\hline
			$P$ & $\dfrac{n_1}{n}$ & $\dfrac{n_2}{n}$ & $...$ & $\dfrac{n_k}{n}$ \\
			\hline
		\end{tabular}
	\end{center}
	\caption{Таблица распределения}
\end{table}

Эмпирическая функция распределения является оценкой, т. е. приближённым значением, генеральной функции распределения

\begin{equation}
	F_n^*(x) \approx F_X(x).
\end{equation}

\subsection{Оценки плотности вероятности}

\subsubsection{Определение}

Оценкой плотности вероятности $f(x)$ называется функция $\widehat{f}(x)$, построенная на основе выборки, приближённо равная $f(x)$

\begin{equation}
	\widehat{f}(x) \approx f(x).
\end{equation}

\subsubsection{Ядерные оценки}

Представим оценку в виде суммы с числом слагаемых, равным объёму выборки:

\begin{equation}
	\widehat{f}_n(x) = \tfrac{1}{nh_n}\sum\limits_{i=1}^n K\left( \tfrac{x - x_i}{h_n} \right).
\end{equation}

Здесь функция $K(u)$, называемая ядерной (ядром), непрерывна и является плотностью вероятности, $x_1, \, ... \: , x_n$ --- элементы выборки, $\{h_n\}$ --- любая последовательность положительных чисел, обладающая свойствами

\begin{equation}
	h_n \underset{n \to \infty}{\longrightarrow} 0; \qquad \dfrac{h_n}{n^{-1}} \underset{n \to \infty}{\longrightarrow} \infty.
\end{equation}

Такие оценки называются непрерывными ядерными \cite[с. 421-423]{b:probSectMath}.\\

Замечание. Свойство, означающее сближение оценки с оцениваемой величиной при $n \rightarrow \infty$ в каком-либо смысле, называется состоятельностью оценки.\\

Если плотность $f(x)$ кусочно-непрерывная, то ядерная оценка плотности является состоятельной при соблюдении условий, накладываемых на параметр сглаживания $h_n$, а также на ядро $K(u)$.\\

Гауссово (нормальное) ядро \cite[с. 38]{a:nonParamRegr}

\begin{equation}
	K(u) = \tfrac{1}{\sqrt{2\pi}}e^{-\tfrac{u^2}{2}}.
\end{equation}

Правило Сильвермана \cite[с. 44]{a:nonParamRegr}

\begin{equation}
	h_n = 1.06\hat{\sigma}n^{-1/5},
\end{equation}

где $\hat{\sigma}$ - выборочное стандартное отклонение.
